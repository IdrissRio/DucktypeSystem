
\subsection{Architettura logica}
\label{sec:logical-arch}
Come già anticipato in Sezione~\ref{sec:func-req},
il sistema dovrà gestire due tipologie di informazioni.\\

La descrizione del grafo principale e le caratteristiche scelte
in fase di configurazione simulano la collocazione dei robot in
un ambiente fisico reale che si mantiene costante: dovranno dunque
essere accessibili dall'intero sistema in qualunque momento,
ma non potranno essere utilizzate per risolvere la ricerca della query.
Avremo un modulo `\textbf{\emph{DSDataFacade}}' che conterrà:
\begin{itemize}
\item \emph{mainGraph}: il grafo caricato dall'utente rappresentante
  l'ambiente totale;
\item \emph{numRobot}: il numero di robot collocati nella mappa;
\item \emph{occupied}: l'elenco dei nodi in cui i robot sono posizionati.
\end{itemize}
I primi due campi della `\emph{DSDataFacade}' sono costanti e vengono
istanziati in fase di inizializzazione: in questa fase si provvede anche
a inizializzare in modo random il vettore `\emph{occupied}'. Quest'ultimo
potrà essere modificato durante l'esecuzione del sistema in seguito allo
spostamento dei robot.
Solo il modulo di visualizzazione potrà accedere alla `\emph{DSDataFacade}'
richiedendo la posizione dei robot, permettendo all'utente di monitorare
il loro movimento: non è infatti ammesso ai singoli robot di conoscere
la posizione degli altri né il loro numero complessivo per risolvere
il problema.\\

Gli altri moduli del sistema sono volti ad implementare i
ruoli coinvolti in un'implementazione fisica reale.
Utilizzando la terminologia di \emph{Akka}, questi costituiscono
\emph{attori}, ovvero astrazioni con comportamento fortemente coeso
che implementano determinate funzionalità innescate dallo scambio di
messaggi. Gli attori si possono trovare su diversi \emph{Actor Systems},
che rappresentano nodi distinti della rete.

Ogni \textbf{robot} è un Actor System su cui sono in esecuzione più attori:
\begin{itemize}
\item `\emph{DSRobot}': è l'attore principale; viene costruito specificando
  il nodo in cui è stato posizionato in fase di inizializzazione.
  Accede alla `\emph{DSDataFacade}' per caricare in memoria la vista
  `\emph{myView}' a partire da `\emph{myNode}'
  del grafo principale, simulando in questo modo l'effettiva
  acquisizione di conoscenza tramite sensori dell'ambiente.

  Quando una query viene sottoposta al sistema, `\emph{DSRobot}' riceve
  un messaggio che impone di creare un attore figlio
  `\emph{DSQueryChecker}'.
  La definizione di un'adeguata gerarchia tra attori si rivela fondamentale
  per poter gestire i fallimenti: infatti il processo padre sarà
  responsabile dei figli.
\item `\emph{DSQueryCheckers}': sono gli attori figli costruiti in un
  robot. Il loro comportamento è estremamente specializzato alla
  verifica e alla gestione di una specifica. Sono presenti in numero pari
  al numero delle query in circolazione in quel momento e vengono
  distrutti dal processo padre appena la query che gestisono ha raggiunto
  un risultato.
\end{itemize}

Ogni \textbf{client} che si interfaccia al sistema costituisce un nuovo Actor System.
In esso c'è un solo attore `\emph{DSClusterInterfaceActor}':
esso deve essere il punto di accesso al cluster per l'utente.\\
In relazione al cluster deve mandare la query che l'utente
sottopone e ottenere da questo il risultato della computazione;
deve essere informato anche degli eventuali fallimenti
riscontrati tra i robot per poter ripresentare la query.\\
Per gestire l'interfaccia con l'utente potrà accedere alle informazioni
presenti in ogni host, contenute nel modulo `\emph{DSCluster}'.
Tale modulo è condiviso con il modulo di visualizzazione grafica
`\emph{DSView}' che si occupa di settare i dati e i comandi dell'utente.
In particolare verrà memorizzato in `\emph{DSCluster}' l'elenco delle
query sottoposte da quel particolare client con i relativi
risultati.

Si noti che tali Actor System non conoscono la posizione degli altri,
dunque non sarà possibile mandare messaggi one-to-one,
ma si dovrà sempre passare per meccanismi di comunicazione che svincolano
dalla necessità di conoscere la località dei destinatari.

\subsection{Protocollo e algoritmo}
\label{sec:protocols}
Communication between components.
UML sequence diagrams go here.
Also, put here a detailed description of distributed algorithms
used to solve specific problems of the project.

\subsection{Architettura fisica e sviluppo}
\label{sec:deploy}
Which nodes and platforms involved, and where each
component is deployed.

