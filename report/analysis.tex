In questa sezione approfondiremo le specifiche del
\emph{subgraph isomorphism problem}; inoltre analizzeremo i
requisiti del sistema che lo implementa applicato a un particolare
contesto fisico.

\subsection{Modellizzazione del problema}
\label{sec:problem}
Tratteremo grafi descritti da una coppia
$\cG = (\Vset, E)$ con $\Vset = \{ v_i \}_{1 \leq i \leq n}$ insieme di
etichette (i nodi del grafo) e $E \sseq \Vset \times \Vset$
insieme di archi non etichettati e non orientati con
$(v_i, v_j) \in E$ se e solo se $v_i$ e $v_j$ sono adiacenti in $\cG$.

In generale, con \emph{subgraph isomorphism problem} si intende
stabilire se un grafo query etichettato $\cQ = (\Vset_\cQ, E_\cQ)$ è
sottografo di un altro grafo etichettato $\cG = ( \Vset, E)$,
ovvero se vale: $\Vset_\cQ \sseq \Vset$ e $E_\cQ \sseq E$.
Nel seguito indicheremo con l'operatore `$\sqsseq$' la relazione
di sottografo.

La versione \emph{distribuita} dell'algoritmo prevede che
non si abbia una conoscenza centralizzata del grafo $\cG$,
ma lo si conosca solo come somma di conoscenza parziali.

\begin{definition}[Vista di un nodo]
Dato un nodo $v_i \in \Vset$, la sua \emph{vista} del grafo
$\cG = (\Vset, E)$ è definita come $\view(v_i) \defeq (\{v_i\}, E_i)$, con
$E_i = \bigl\{\,(v_j, v_k) \in E \bigm| k = i \vee j = i \,\bigr\}$.
\end{definition}

Per ogni insieme di nodi $\Vset' \sseq \Vset$ si ha che
$\bigcup_{v \in \Vset'} \view(v) \sqsseq \cG$.
L'unione delle viste di un insieme di nodi $\Vset'$ equivale
all'intero grafo solo nel caso $\Vset'$ costituisca il suo
\emph{vertex cover}.\\

Il sistema viene applicato a grafi che rappresentano la
disposizione spaziale di alcuni luoghi (i nodi etichettati)
connessi da particolari archi (non etichettati).
I robot distribuiti nella mappa sono in grado di acquisire
l'intera vista del grafo a partire dal nodo in cui si trovano:
se $\cR$ è la conoscenza del robot posizionato sul nodo $v$,
vale dunque sempre che $\view(v) \sqsseq \cR$.

\subsection{Requisiti funzionali}
\label{sec:func-req}
Il sistema deve implementare l'algoritmo e fornire adeguati strumenti
per monitorare la sua esecuzione.
La componente centrale sarà formata dai processi dei vari robot e dai
processi dei client che sottopongono le query; a questa
sarà affiancata una componente atta a \emph{simulare}
l'ambiente, ovvero la mappa nella quale i robot sono collocati,
raccogliendo informazioni per controllare la correttezza dei risultati
restituiti.

All'utente si propone un'interfaccia grafica che in
fase di inizializzazione richiede:
\begin{itemize}
\item un file contenente la mappa totale in formato
      \texttt{.DOT, .DGS, .GML, .TLP, .NET, .graphML, .GEXF}.
\item il numero $n$ di robot da collocare nella mappa.
\end{itemize}
Questo consentirà all'utente di visualizzare la mappa e la posizione,
definita casualmente, dei robot inseriti. Si noti che le
informazioni complessive e centralizzate riguardanti la collocazione
dei robot e il grafo complessivo \emph{non} potranno essere
utilizzate per risolvere il problema, né saranno rese disponibili
all'utente dell'applicazione reale: hanno infatti come unico scopo
quello di facilitare la visualizzazione e il controllo di
questa simulazione.

Nella fase di inizializzazione ogni robot acquisisce la vista del
grafo a partire dalla sua posizione;
successivamente l'utente potrà sottoporre query nello stesso formato
del grafo e far partire la ricerca.
Il sistema può ricevere un numero arbitrario di query
contemporaneamente e deve provvedere a versionarle.

L'interfaccia permette di simulare la connessione di un altro
client che accede allo stesso insieme di robot, sottoponendo
le proprie query. Ogni client host deve essere indipendente:
l'accesso è dunque identico per ogni host come se fosse l'unico a
interrogare il cluster.

Quando la computazione termina, il client che ha sottoposto la
query riceve l'esito. Questo può essere:
\begin{itemize}
\item \texttt{MATCH}: nel caso in cui la somma delle conoscenze
parziale dei robot è in grado di coprire tutta la query: ovvero quando \[
\cQ \sqsseq \bigcup_{1 \leq i \leq n} \cR_i, \]
con $\cR_i$ conoscenza parziale del robot $i$;
\item \texttt{FAIL}: nel caso in cui vengano riscontrate inconsistenze
tra la somma delle conoscenze parziali (ma localmente certe)
dei sensori e la query: ovvero quando un arco $(x, y) \in E_\cQ$
non è presente in $E_\cR$ con $\cR$ la conoscenza del robot posizionato
sul nodo $x$ (infatti è garantito che $\view(x) \sqsseq \cR$);
\item \texttt{DONTKNOW}: in tutti gli altri casi, ovvero quando
data la mancanza di una visione totale del grafo,
non si riesce a determinare in maniera effettiva né l'esistenza né
il fallimento della query.
\end{itemize}

In caso di \texttt{DONTKNOW} l'utente può imporre ai robot di muoversi
nel grafo, cercando in questo modo di acquisire nuova conoscenza,
e in seguito ripresentare la stessa query.
I robot devono dunque scegliere autonomamente un nuovo nodo,
tra i propri adiacenti e caricare in memoria la vista del nuovo
nodo in cui si posizionano: poiché in un contesto reale i
robot hanno limitate capacità di memorizzazione,
in questo processo verranno scordati i nodi conosciuti due
passi indietro.

I robot devono poter comunicare esclusivamente tramite scambio di
messaggi: la loro capacità di memorizzazione deve essere sufficiente
a mantenere almeno l'intera query sottoposta.\\


\begin{figure}
\centering
{\begin{pdfpic}
\psset{xunit=1cm,yunit=1cm,runit=1cm}
\psset{origin={0,0}}
\pspicture*[](-0.7,-0.5)(9,2.5)
\rput(-0.5,1){$\mathcal{G}$}
\begin{green}\rput(1.3,0.7){$\mathcal{R}$}\end{green}
\psline[linecolor=black](0,2)(1,1)
\psline[linecolor=black](2,2)(1,1)
\psline[linecolor=black](0,0)(1,1)
%% \psline[linecolor=black](2,0)(1,1)
\psline[linecolor=black](2,2)(3.5,1)
\psline[linecolor=black](2,0)(3.5,1)
\psline[linecolor=black](2,2)(0,2)
\psline[linecolor=black](1,1)(3.5,1)
\psline[linecolor=green, linestyle=dashed](0,2)(1,1)
\psline[linecolor=green, linestyle=dashed](2,2)(1,1)
\psline[linecolor=green, linestyle=dashed](0,0)(1,1)
\psline[linecolor=green, linestyle=dashed](1,1)(3.5,1)
%% \psline[linecolor=blue, linestyle=dashed](2,0)(1,1)
\pscircle*[linecolor=black](0, 2){2pt}
\pscircle*[linecolor=black](2, 2){2pt}
\pscircle*[linecolor=green](1, 1){2pt}
\pscircle*[linecolor=black](3.5, 1){2pt}
\pscircle*[linecolor=black](0, 0){2pt}
\pscircle*[linecolor=black](2, 0){2pt}
\begin{footnotesize}
\rput(0,2.3){$a$}
\rput(2,2.3){$b$}
\rput(0.7,1){$c$}
\rput(3.7,1.2){$d$}
\rput(0,-0.3){$e$}
\rput(2,-0.3){$f$}
\end{footnotesize}
%%
\psline[linecolor=blue](7,2)(6,1)
\psline[linecolor=blue](5,0)(6,1)
\psline[linecolor=yellow](7,2)(8.5,1)
\psline[linecolor=red, linestyle=dashed](6,1)(7.5,0)
\pscircle*[linecolor=yellow](7, 2){2pt}
\pscircle*[linecolor=blue](6, 1){2pt}
\pscircle*[linecolor=yellow](8.5, 1){2pt}
\pscircle*[linecolor=blue](5, 0){2pt}
\pscircle*[linecolor=white](7.5, 0){2pt}
\pscircle[linecolor=red](7.5, 0){2pt}
\begin{footnotesize}
\rput(7,2.3){$b$}
\rput(5.7,1){$c$}
\rput(8.7,1.2){$d$}
\rput(5,-0.3){$e$}
\rput(7.5,-0.3){$f$}
\end{footnotesize}
%%
\endpspicture
\end{pdfpic}
}
\caption{Nella parte sinistra: un grafo etichettato $\cG$
con un robot posizionato sul nodo $c$. Nella parte destra: una query
$\cQ$ sottoposta al sistema.}
\label{fig:graph-query}
\end{figure}

\begin{example}
Si consideri il grafo $\cG = (\Vset, E)$ presentato nella parte sinistra
della Figura~\ref{fig:graph-query}. Si ha:~\footnote{Nell'insieme di archi
riportiamo una sola coppia $(v_i, v_j)$, lasciando implicita la sua
opposta $(v_j, v_i)$.}
\[
\Vset = \bigl\{\,a, b, c, d, e, f\,\bigr\}, \ \,
E = \bigl\{\,(a,b),(a,c),(b,c),(b,d),(c,e),(c,d),(d,f)\,\bigr\};
\]
la conoscenza parziale del grafo,
evidenziata in figura con linee verdi tratteggiate,
è la vista acquisita dal robot collocato
sul nodo $c$: \[
\cR = \view(c) = \left(\bigl\{c\bigr\},\,
\bigl\{\,(a,c), (b,c), (c,e), (c,d)\,\bigr\}\right). \]
Si sottopone la query $\cQ = (\Vset_\cQ, E_\cQ)$ con:\[
\Vset_\cQ = \bigl\{\,b, c, d, e, f\,\bigr\}, \ \,
E_\cQ = \bigl\{\,(b,c),(b,d),(c,e),(c,f)\,\bigr\}. \]

Poiché esiste un arco $(c,f) \in E_\cQ$ che non è presente in $E_\cR$
con il robot posizionato in $c$, esso potrebbe affermare immediatamente
che tale arco non è presente nel grafo principale
(in quanto la sua conoscenza è localmente certa): il risultato della query
è \texttt{FAIL}.

Considerando una query $\cQ'$ ottenuta da $\cQ$ rimuovendo l'arco
$(c, f)$, il robot può verificare la presenza degli archi $(b,c)$ e $(c,e)$
(marcati in blu in figura), ma non può affermare nulla sui restanti:
siamo dunque nel caso in cui $\cQ' \not\sqsseq \cR$ e il risultato
è \texttt{DONTKNOW}.
L'utente allora può imporre il movimento del robot e ritentare
la query ancora da verificare (marcata in giallo in figura).
Supponiamo che il robot si sposti sul nodo $b$: esso può allora verificare
l'esistenza degli archi rimanenti e concludere con \texttt{MATCH}.
\end{example}

\subsection{Requisiti non funzionali}
\label{sec:nonfunc-req}

\subsubsection*{Scalabilità e trasparenza:}
I robot costituiscono una rete peer-to-peer,
senza alcuna suddivisione di ruoli a priori; i ruoli centralizzati
sono definiti staticamente con il compito fissato di fornire un punto
di accesso al cluster in ingresso (proponendo una nuova query) e
in uscita (raccogliendo informazioni relative alla riuscita
o al fallimento della query).

Un requisito fondamentale è che
la \emph{conoscenza} del grafo complessivo sia totalmente distribuita.
Il sistema deve essere dunque scalabile e flessibile:
si devono poter far spostare i robot senza la necessità di
riconfigurare nulla. L'implementazione delle
comunicazioni tra i punti di accesso centrali al cluster e
tra gli elementi stessi del cluster
deve scalare con il numero di robot ed essere totalmente
svincolata dal conoscere la loro posizione.
Il grado di \emph{location transparency} offerto
deve dunque essere molto elevato.

\subsubsection*{Failure model:}
Di fronte a un fallimento, sia dovuto a una rete inaffidabile,
sia dovuto alla morte di un processo in esecizione in esso,
l'approccio è quello di ristabilire la
condizione del sistema senza che
l'utente ne venga informato. Eventualmente quindi il sistema deve poter
recuperare e ripresentare in modo automatico le query che si stavano
verificando. In caso di morte di un robot però,
essendo un fallimento critico, l'utente di ogni client host viene informato,
(anche se l'host non ha query attive in quel momento):
nella simulazione del sistema si provvede immediatamente a
sostituire il robot con uno nuovo posizionato nello stesso luogo.

\subsubsection*{Correttezza del risultato:}
Il sistema deve riuscire a rispondere con un messaggio di
\texttt{MATCH} / \texttt{FAIL} / \texttt{DONTKNOW} correttamente
rispetto a quanto definito nelle specifiche funzionali.
Anche nel caso di fallimenti, si garantisce che
il sistema non dà falsi \texttt{MATCH} o falsi \texttt{FAIL} della query:
viene ammessa solo la restituzione di un falso \texttt{DONTKNOW}.

Infatti,
in caso un robot muoia durante la verifica, può accadere che il sistema,
sebbene la somma delle conoscenze parziali sia in grado di definire
un risultato, risponda comunque \texttt{DONTKNOW} all'utente.

Quando la conoscenza dei sensori aumenta,
non è garantito che le nuove informazioni possano essere immediatamente
utilizzate se queste sono state acquisite \emph{durante} la fase di verifica
di una query: anche in questo caso, il sistema può rispondere eventualmente
\texttt{DONTKNOW}, anche quando l'insieme (aggiornato) delle conoscenze
parziali dei sensori sarebbe in grado di dare una risposta più specifica.

