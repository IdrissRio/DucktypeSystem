\subsection{Struttura delle directory dei sorgenti}
I file sorgenti \emph{.java} del progetto si possono trovare nella directory:

\printlog{DS-BecchiRiouak/project/src/main/java/it/uniud/ducktypesystem/}

In essa è contenuta la main class (\emph{DucktypeSystem.java})
ed è strutturata in sottocartelle come segue:
\begin{itemize}
\item controller : contiene la classe DSApplication.java punto di accesso
  all'applicazione MVC, e le classi del sistema di Log.
\item view : contiene le classi che implementano il modulo di visualizzazione grafica.
\item distributed : contiene le classi che implementano i processi e le strutture
  dati del sistema distribuito.
\end{itemize}

La cartella ``distributed'' è a sua volta strutturata in:
\begin{itemize}
\item system : contiene la classe volta a simulare l'esistenza di un
  contesto fisico reale (\emph{DSDataFacade})
  e la classe che simula le connessioni di diversi host
  fornendo loro punti di accesso al sistema (\emph{DSCluster}).
\item data : contiene le classi che implementano le strutture dati
  utilizzate dal sistema.
\item impl : contiene le classi che implementano gli attori del sistema.
\item messages : contiene le classi utilizzate come messaggi.
\item errors : contiene le classi utilizzate per segnalare o simulare
  particolari comportamentei anomali del sistema.
\end{itemize}

\subsection{Materiale per il testing}
La cartella \emph{DS-BecchiRiouak/test/} contiene un insieme di file che
rappresentano grafi nel formato richiesto dal sistema.
Per ogni file contenente un grafo principale sono associati (almeno)
altri due file contenenti possibili query da sottoporre: la notazione
usata è:
\begin{itemize}
\item NOME-G.DGS : per il grafo principale.
\item NOME-Qmatch.DGS : per le query che, se sottoposte al grafo ``NOME'',
  hanno risultato atteso \texttt{MATCH}.
\item NOME-Qfail.DGS : per le query che, se sottoposte al grafo ``NOME'',
  hanno risultato atteso \texttt{FAIL}.
\end{itemize}

Generalmente, sottoporre una query a un grafo diverso da quello riportato
nel nome del file conduce al risultato \texttt{FAIL}.

Si noti che per ottenere il risultato atteso può essere necessario
che la query venga ripresentata più volte dopo aver fatto muovere i robot:
nei tentativi intermedi il risultato può infatti essere ancora \texttt{DONTKNOW}.

Sono stati scritti anche alcuni test per il framework \emph{Junit}, volti al
testing delle strutture dati e dell'algoritmo \texttt{check-and-reduce()}.
Questi si possono trovare nella cartella:
\printlog{DS-BecchiRiouak/project/src/test/java/it/
  uniud/ducktypesystem/tests/data/}

\subsection{Run e testing del progetto}
Il progetto è eseguibile utilizzando lo script

\begin{quote}\texttt{
    \$ cd DS-BecchiRiouak\\
    \$ /dsquack.sh }
\end{quote}
che ammette l'inserimento dei parametri:
\begin{itemize}
\item quiet : (di default) reindirizza i messaggi di log interni in un file nella
  cartella \emph{DS-BecchiRiouak/log/ }
\item noisy : stampa i messaggi di log interni in std output.
\end{itemize}

Tramite l'interfaccia grafica è possibile settare le configurazioni di sistema:
il grafo principale può essere selezionato tra i grafi della cartella
\emph{DS-BecchiRiouak /test}.
Si vedrà allora una rappresentazione grafica della mappa caricata con
evidenziati i nodi occupati dai robot.

Il log del sistema annuncerà:
\begin{quote}\texttt{
 <date time>: DucktypeSystem v 0.1\\
 <date time>: Robot posizionati in: `node' `node' `node'\\
 <date time>: INFO: starting the AKKA environment.\\
 <date time>: AKKA: Every Robot is connected
 }
\end{quote}

Se tale fase è andata a buon fine, si potrà sottoporre una query tramite
l'interfaccia grafica: questa permette di selezionare l'host tramite selezione
tra gli indici degli host correntemente attivi (in questa fase è anche possibile
connettere un nuovo host).

Le query sottoposte verranno elencate identificandole per nome e per l'indice
numerico dell'host da cui provengono, insieme al loro stato:
\begin{quote}\texttt{
    NEW / MATCH / FAIL / DONTKNOW
    }
\end{quote}

In caso di \texttt{DONTKNOW} verrà anche riportata la parte della query che è
rimasta ancora da verificare.

Il menu di setting propone le opzioni relative al setting delle probabilità
di fallimento dei vari processi.

\subsection{Messaggi di Log interni}
I messaggi di log interni riportano informazioni riguardanti lo stato di
esecuzione: riportiamo i messaggi prodotti dai principali passi di
esecuzione del sistema, utili per monitorarne la correttezza.

La creazione del cluster e il JOIN di un nuovo Actor System produrrà messaggi di log
che riportano l'indirizzo e la porta del nuovo Actor System.

\printlog{
  [INFO] [..omissis..] Starting remoting\\
  [INFO] [..omissis..] Remoting started; listening on addresses :[akka.tcp://ClusterSystem@127.0.0.1:2551]\\
  [INFO] [..omissis..] Cluster Node [akka.tcp://ClusterSystem@127.0.0.1:2551] - Starting up...\\
  [INFO] [..omissis..] Cluster Node [akka.tcp://ClusterSystem@127.0.0.1:2551] - Registered cluster JMX MBean [akka:type=Cluster,port=2551]\\
  [INFO] [..omissis..] Cluster Node [akka.tcp://ClusterSystem@127.0.0.1:2551] - Started up successfully\\
  [INFO] [..omissis..] Cluster Node [akka.tcp://ClusterSystem@127.0.0.1:2551] - Metrics collection has started successfully}

La creazione di un nuovo attore DSRobot produrrà messaggi di log che riportano
il nome assegnato al robot, l'etichetta del nodo sul quale è stato posizionato
e una descrizione della vista acquisita.
\printlog{[INFO] [..omissis..] [akka://ClusterSystem/user/ROBOT] `RobotName' created on `myNode' with view Graph: [...]}


L'invio di una nuova query da parte di un host comporta la creazione di nuovi attori DSQueryChecker,
uno per ogni robot: ciò produrrà messaggi di log che riportano
l'etichetta del nodo sul quale sono posizionati e la vista ereditata dal padre.
\printlog{[INFO] [..omissis..] [akka://ClusterSystem/user/ROBOT/`queryId'] QueryChecker created on `node' for query `queryId'}


Quando un DSRobot viene informato della morte di un suo DSQueryChecker
genera messaggi di log riportanti lo stato con il quale l'attore è terminato.

In caso di morte in stato DONE o in seguito a una normale terminazione del
processo, ci si aspetta di ottenere messaggi della forma:
\printlog{[INFO] [..omissis..] [akka://ClusterSystem/user/ROBOT/`queryId'] QueryChecker DEATH in DONE. }

In caso di morte in WAITING ci si aspetta di ottenere nuovamente il messaggio
che testimonia la creazione di un nuovo DSQueryChecker sullo stesso path.
\printlog{
  [INFO] [..omissis..] [akka://ClusterSystem/user/ROBOT/`queryId'] QueryChecker DEATH in DONE.\\
  [...]\\
  [INFO] [..omissis..] [akka://ClusterSystem/user/ROBOT] Recreating QueryChecker...\\
  [...]\\
  [INFO] [..omissis..] [akka://ClusterSystem/user/ROBOT/`queryId'] QueryChecker created on `node' for query `queryId'}

In caso di morte in CRITICALWORK ci si aspetta di ottenere messaggi di log
che testimonino la terminazione della query e la richiesta di un nuovo
tentativo al DSClusterInterfaceActor.
\printlog{
  [INFO] [..omissis..] [akka://ClusterSystem/user/ROBOT/`queryId'] QueryChecker DEATH in CRITICAL WORK.\\
  [...]\\
  [INFO] [..omissis..] [akka://ClusterSystem/user/CLUSTERMANAGERX] Retrying query...}

L'inoltro di una query appena elaborata da parte di un DSQueryChecker produce
messaggi di log della forma:
\printlog{
  [INFO] [..omissis..] [akka://ClusterSystem/user/ROBOT/`queryId'] FORWARDING from `node': query: [...]
  }

Il raggiungimento di un risultato produce messaggi della forma:

\printlog{
  [INFO] [..omissis..] [akka://ClusterSystem/user/ROBOT/`queryId'] MATCH da: `node'\\
  [INFO] [..omissis..] [akka://ClusterSystem/user/ROBOT/`queryId'] FAIL da: `node'\\
  [INFO] [..omissis..] [akka://ClusterSystem/user/ROBOT/`queryId'] DONTKNOW da: `node'}
La morte di un DSRobot è simulata tramite il lancio di un'eccezione dedicata. Si otterrà dunque:
\printlog{
  [ERROR] [..omissis..] [akka://ClusterSystem/user/ROBOT] Robot2 DIED.
  it.uniud.ducktypesystem.distributed.errors.DSSystemFailureSimulation: Robot2 DIED.
  at [...]}

Ci si aspetta di leggere il messaggio che testimonia la creazione di un nuovo robot
e i messaggi che impongono la terminazione di tutte le query attive.
\printlog{
  [INFO] [..omissis..] [akka://ClusterSystem/user/ROBOT] `Robotname' killing all my sons.}

La terminazione del sistema produce numerosi messaggi di log, tra questi
osserveremo anche messaggi che riportano un atteso "dead letter encountered".
